%%%%%%%%%%%%%%%%%%%% author.tex %%%%%%%%%%%%%%%%%%%%%%%%%%%%%%%%%%%
%
% sample root file for your "contribution" to a contributed volume
%
% Use this file as a template for your own input.
%
%%%%%%%%%%%%%%%% Springer %%%%%%%%%%%%%%%%%%%%%%%%%%%%%%%%%%


% RECOMMENDED %%%%%%%%%%%%%%%%%%%%%%%%%%%%%%%%%%%%%%%%%%%%%%%%%%%
\documentclass[graybox]{svmult}

% choose options for [] as required from the list
% in the Reference Guide

\usepackage{mathptmx}       % selects Times Roman as basic font
\usepackage{helvet}         % selects Helvetica as sans-serif font
\usepackage{courier}        % selects Courier as typewriter font
\usepackage{type1cm}        % activate if the above 3 fonts are
                            % not available on your system
%
\usepackage{makeidx}         % allows index generation
\usepackage{graphicx}        % standard LaTeX graphics tool
                             % when including figure files
\usepackage{multicol}        % used for the two-column index
\usepackage[bottom]{footmisc}% places footnotes at page bottom
\usepackage{newunicodechar}

% see the list of further useful packages
% in the Reference Guide

\makeindex             % used for the subject index
                       % please use the style svind.ist with
                       % your makeindex program

%%%%%%%%%%%%%%%%%%%%%%%%%%%%%%%%%%%%%%%%%%%%%%%%%%%%%%%%%%%%%%%%%%%%%%%%%%%%%%%%%%%%%%%%%

\begin{document}

\title*{eMDPM: EFFICIENT MULTI-DIMENSIONAL
PATTERN MATCHING ALGORITHM FOR GPU}
\titlerunning{eMDPM}
% Use \titlerunning{Short Title} for an abbreviated version of
% your contribution title if the original one is too long
\author{Supragya Raj$^{\alpha}$, Siddha Prabhu Chodnekar$^{\alpha}$ , Harish T$^{\alpha}$ and Harini Sriraman$^{\alpha}$}
\authorrunning{Supragya Raj, S.P Chodnekar, Harish T, Harini S.}
% Use \authorrunning{Short Title} for an abbreviated version of
% your contribution title if the original one is too long
\institute{Supragya Raj \at SCSE, VIT Chennai$^{\alpha}$, \email{supragyaraj@gmail.com}
\and Siddha Prabhu Chodnekar$^{\alpha}$, Harish T$^{\alpha}$, Harini Sriraman$^{\alpha}$, \newline\email{sprabhu.chodnekar@vit.ac.in}, \email{harini.s@vit.ac.in}
\newline\newline
${\alpha}$: School of Computing Science and Engineering, Vellore Institure of Technology, Chennai Campus, Chennai
600127, India}
%
% Use the package "url.sty" to avoid
% problems with special characters
% used in your e-mail or web address
%
\maketitle

\abstract*{Each chapter should be preceded by an abstract (10--15 lines long) that summarizes the content. The abstract will appear \textit{online} at \url{www.SpringerLink.com} and be available with unrestricted access. This allows unregistered users to read the abstract as a teaser for the complete chapter. As a general rule the abstracts will not appear in the printed version of your book unless it is the style of your particular book or that of the series to which your book belongs.
Please use the 'starred' version of the new Springer \texttt{abstract} command for typesetting the text of the online abstracts (cf. source file of this chapter template \texttt{abstract}) and include them with the source files of your manuscript. Use the plain \texttt{abstract} command if the abstract is also to appear in the printed version of the book.}

\abstract{Parallelizing pattern matching in multi-dimensional images is very vital in many applications to improve performance. With SIMT architectures, the performance can be greatly enhanced if the hardware threads are utilized to the maximum. In case of pattern matching algorithms, the main bottle neck arises due to the reduction operation that needs to be performed on the multiple parallel search operations. This can be solved by using Shift-Or operations. Recent trend has shown the improvement in pattern matching using Shift-Or operations for bit pattern matching. This has to be extended for multiple dimensional images like hyper-cubes. In this paper, we have extended the shift-or pattern matching for multi-dimensional images. The algorithm is implemented for GPU architectures. The complexity of the proposed algorithm is $ m*\frac{log(n)}{kw} $ where $m$ is the number of dimensions, $n$ is the size of the array if the multidimensional matrix values are placed in a single dimensional array, $k$ is the size
of the pattern and w is the size of the tile. From the result analysis it is found that the performance is maximum, when the pattern size matches the tile size and it is less than 64. This restriction is due to the size of the warp considered.}

\section{Introduction}
\label{sec:1}
With advent of GPUs, execution time of image processing applications has greatly reduced. Pattern searching is an integral part of many applications. Any improvement in the performance of pattern matching of multi-dimensional image will greatly improve the performance of many GPU based applications. Pattern matching can be broadly classified into exact bit pattern matching and approximate bit pattern matching algorithms. The main bottle neck with respect to improving the performance pattern searching is reduction operation. In recent times, shift–or operation is used to improve the efficiency of reduction operation based pattern matching algorithms.

Parallel pattern matching involves tasks identification, communication identification, task agglomeration and mapping of tasks to processors. GPU is a SIMT architecture that can be utilized effectively for pattern matching algorithms. The matching of individual patterns can happen in the stream processors available inside the stream multiprocessors. The communication involves the reduction process. Agglomeration and mapping depends on the number of processors available. If we consider $n$ to be the number of primitive tasks and $p$ to be the number of processors, then $\frac{n}{p}$ tasks can be assigned per processor.
\section{Related Work}
\label{sec:2}
% Always give a unique label
% and use \ref{<label>} for cross-references
% and \cite{<label>} for bibliographic references
% use \sectionmark{}
% to alter or adjust the section heading in the running head
Pattern matching involves the process of identifying all patterns in a given text. There are many practical applications for pattern matching algorithms. The pattern matching algorithms can be classified with respect to the dimensions of the data. Further classification of these algorithms include the type of pattern matching namely, exact pattern
matching and approximate pattern matching.

\subsection{Exact and approximate pattern matching}
\label{subsec:2}
Exact pattern matching provides a solution by viewing each text position to be a possible pattern start. For exact pattern matching algorithm, the complexity is O(mn) where m*n is the dimension of the text matrix. Many improvements exist in the literature for exact pattern matching algorithms. Usage of linear automata [1] for the purpose is the most popular technique. This reduces the complexity of the pattern match-
ing algorithm to be O(m+n). Good-suffix heuristic algorithm proposed in [2] provides an improved performance of O(n) with constant space requirement. In [3], bad character rule strategy is generalized and improves the performance of the pattern matching algorithm with time complexity of O(n/m). Simple pattern searching algorithm proposed in [4] improves the average time complexity to be O(m+n).

\begin{quotation}
Please do not use quotation marks when quoting texts! Simply use the \verb|quotation| environment -- it will automatically render Springer's preferred layout.
\end{quotation}


\subsubsection{Subsubsection Heading}
Instead of simply listing headings of different levels we recommend to
let every heading be followed by at least a short passage of text.
Further on please use the \LaTeX\ automatism for all your
cross-references and citations as has already been described in
Sect.~\ref{subsec:2}, see also Fig.~\ref{fig:1}\footnote{If you copy
text passages, figures, or tables from other works, you must obtain
\textit{permission} from the copyright holder (usually the original
publisher). Please enclose the signed permission with the manuscript. The
sources\index{permission to print} must be acknowledged either in the
captions, as footnotes or in a separate section of the book.}

Please note that the first line of text that follows a heading is not indented, whereas the first lines of all subsequent paragraphs are.

% For figures use
%



\paragraph{Paragraph Heading} %
Instead of simply listing headings of different levels we recommend to
let every heading be followed by at least a short passage of text.
Further on please use the \LaTeX\ automatism for all your
cross-references and citations as has already been described in
Sect.~\ref{sec:2}.

Please note that the first line of text that follows a heading is not indented, whereas the first lines of all subsequent paragraphs are.

For typesetting numbered lists we recommend to use the \verb|enumerate| environment -- it will automatically render Springer's preferred layout.

\begin{enumerate}
\item{Livelihood and survival mobility are oftentimes coutcomes of uneven socioeconomic development.}
\begin{enumerate}
\item{Livelihood and survival mobility are oftentimes coutcomes of uneven socioeconomic development.}
\item{Livelihood and survival mobility are oftentimes coutcomes of uneven socioeconomic development.}
\end{enumerate}
\item{Livelihood and survival mobility are oftentimes coutcomes of uneven socioeconomic development.}
\end{enumerate}


\subparagraph{Subparagraph Heading} In order to avoid simply listing headings of different levels we recommend to let every heading be followed by at least a short passage of text. Use the \LaTeX\ automatism for all your cross-references and citations as has already been described in Sect.~\ref{sec:2}, see also Fig.~\ref{fig:2}.

For unnumbered list we recommend to use the \verb|itemize| environment -- it will automatically render Springer's preferred layout.

\begin{itemize}
\item{Livelihood and survival mobility are oftentimes coutcomes of uneven socioeconomic development, cf. Table~\ref{tab:1}.}
\begin{itemize}
\item{Livelihood and survival mobility are oftentimes coutcomes of uneven socioeconomic development.}
\item{Livelihood and survival mobility are oftentimes coutcomes of uneven socioeconomic development.}
\end{itemize}
\item{Livelihood and survival mobility are oftentimes coutcomes of uneven socioeconomic development.}
\end{itemize}


\runinhead{Run-in Heading Boldface Version} Use the \LaTeX\ automatism for all your cross-references and citations as has already been described in Sect.~\ref{sec:2}.

\subruninhead{Run-in Heading Italic Version} Use the \LaTeX\ automatism for all your cross-refer\-ences and citations as has already been described in Sect.~\ref{sec:2}\index{paragraph}.
% Use the \index{} command to code your index words
%
% For tables use
%
\begin{table}
\caption{Please write your table caption here}
\label{tab:1}       % Give a unique label
%
% Follow this input for your own table layout
%
\begin{tabular}{p{2cm}p{2.4cm}p{2cm}p{4.9cm}}
\hline\noalign{\smallskip}
Classes & Subclass & Length & Action Mechanism  \\
\noalign{\smallskip}\svhline\noalign{\smallskip}
Translation & mRNA$^a$  & 22 (19--25) & Translation repression, mRNA cleavage\\
Translation & mRNA cleavage & 21 & mRNA cleavage\\
Translation & mRNA  & 21--22 & mRNA cleavage\\
Translation & mRNA  & 24--26 & Histone and DNA Modification\\
\noalign{\smallskip}\hline\noalign{\smallskip}
\end{tabular}
$^a$ Table foot note (with superscript)
\end{table}
%
\section{Section Heading}
\label{sec:3}
% Always give a unique label
% and use \ref{<label>} for cross-references
% and \cite{<label>} for bibliographic references
% use \sectionmark{}
% to alter or adjust the section heading in the running head
Instead of simply listing headings of different levels we recommend to
let every heading be followed by at least a short passage of text.
Further on please use the \LaTeX\ automatism for all your
cross-references and citations as has already been described in
Sect.~\ref{sec:2}.

Please note that the first line of text that follows a heading is not indented, whereas the first lines of all subsequent paragraphs are.

If you want to list definitions or the like we recommend to use the Springer-enhanced \verb|description| environment -- it will automatically render Springer's preferred layout.

\begin{description}[Type 1]
\item[Type 1]{That addresses central themes pertainng to migration, health, and disease. In Sect.~\ref{sec:1}, Wilson discusses the role of human migration in infectious disease distributions and patterns.}
\item[Type 2]{That addresses central themes pertainng to migration, health, and disease. In Sect.~\ref{subsec:2}, Wilson discusses the role of human migration in infectious disease distributions and patterns.}
\end{description}

\subsection{Subsection Heading} %
In order to avoid simply listing headings of different levels we recommend to let every heading be followed by at least a short passage of text. Use the \LaTeX\ automatism for all your cross-references and citations citations as has already been described in Sect.~\ref{sec:2}.

Please note that the first line of text that follows a heading is not indented, whereas the first lines of all subsequent paragraphs are.

\begin{svgraybox}
If you want to emphasize complete paragraphs of texts we recommend to use the newly defined Springer class option \verb|graybox| and the newly defined environment \verb|svgraybox|. This will produce a 15 percent screened box 'behind' your text.

If you want to emphasize complete paragraphs of texts we recommend to use the newly defined Springer class option and environment \verb|svgraybox|. This will produce a 15 percent screened box 'behind' your text.
\end{svgraybox}


\subsubsection{Subsubsection Heading}
Instead of simply listing headings of different levels we recommend to
let every heading be followed by at least a short passage of text.
Further on please use the \LaTeX\ automatism for all your
cross-references and citations as has already been described in
Sect.~\ref{sec:2}.

Please note that the first line of text that follows a heading is not indented, whereas the first lines of all subsequent paragraphs are.

\begin{theorem}
Theorem text goes here.
\end{theorem}
%
% or
%
\begin{definition}
Definition text goes here.
\end{definition}

\begin{proof}
%\smartqed
Proof text goes here.
\qed
\end{proof}

\paragraph{Paragraph Heading} %
Instead of simply listing headings of different levels we recommend to
let every heading be followed by at least a short passage of text.
Further on please use the \LaTeX\ automatism for all your
cross-references and citations as has already been described in
Sect.~\ref{sec:2}.

Note that the first line of text that follows a heading is not indented, whereas the first lines of all subsequent paragraphs are.
%
% For built-in environments use
%
\begin{theorem}
Theorem text goes here.
\end{theorem}
%
\begin{definition}
Definition text goes here.
\end{definition}
%
\begin{proof}
\smartqed
Proof text goes here.
\qed
\end{proof}
%
\begin{acknowledgement}
If you want to include acknowledgments of assistance and the like at the end of an individual chapter please use the \verb|acknowledgement| environment -- it will automatically render Springer's preferred layout.
\end{acknowledgement}
%
\section*{Appendix}
\addcontentsline{toc}{section}{Appendix}
%
%
When placed at the end of a chapter or contribution (as opposed to at the end of the book), the numbering of tables, figures, and equations in the appendix section continues on from that in the main text. Hence please \textit{do not} use the \verb|appendix| command when writing an appendix at the end of your chapter or contribution. If there is only one the appendix is designated ``Appendix'', or ``Appendix 1'', or ``Appendix 2'', etc. if there is more than one.

\begin{equation}
a \times b = c
\end{equation}

%%%%%%%%%%%%%%%%%%%%%%%%% referenc.tex %%%%%%%%%%%%%%%%%%%%%%%%%%%%%%
% sample references
% %
% Use this file as a template for your own input.
%
%%%%%%%%%%%%%%%%%%%%%%%% Springer-Verlag %%%%%%%%%%%%%%%%%%%%%%%%%%
%
% BibTeX users please use
% \bibliographystyle{}
% \bibliography{}
%

\begin{thebibliography}{99.}%
% and use \bibitem to create references.
%
% Use the following syntax and markup for your references if 
% the subject of your book is from the field 
% "Mathematics, Physics, Statistics, Computer Science"
%
% Contribution 

\bibitem{psysoc-journal} Mitani, Y., Ino, F. and Hagihara, K., 2017. Parallelizing exact and approximate string matching via inclusive scan on a GPU. \textit{IEEE Transactions on Parallel and Distributed Systems}, 28(7), pp.1989-2002 
\bibitem{psysoc-journal} Agrawal, J., Diao, Y., Gyllstrom, D. and Immerman, N., 2008, June. Efficient pattern matching over event streams. In \textit{Proceedings of the 2008 ACM SIGMOD international conference on Management of data} (pp. 147-160). ACM.
%
\bibitem{psysoc-journal} Cantone, D., Cristofaro, S. and Faro, S., 2010, August. A Space-Efficient Implementation of the Good-Suffix Heuristic. In \textit{Stringology} (pp. 63-75).
%
% Journal article by DOI
\bibitem{psysoc-DOI}Sahli, M. and Shibuya, T., 2012. Max-shift BM and max-shift horspool: Practical fast exact string matching algorithms. \textit{Journal of information processing}, 20(2), pp.419-425.

\bibitem{psysoc-DOI}Park, B. and Won, C.S., 2014, June. Fast binary matching for edge histogram descriptor. In \textit{Consumer Electronics (ISCE 2014), The 18th IEEE International Symposium} (pp. 1-2).
\bibitem{psysoc-DOI}Hirvola, T. and Tarhio, J., 2017. Bit-Parallel Approximate Matching of Circular Strings with k Mismatches. \textit{Journal of Experimental Algorithmics (JEA)}, 22, pp.1-5.
\bibitem{psysoc-DOI}Pfaffe, P., Tillmann, M., Lutteropp, S., Scheirle, B. and Zerr, K., 2016, August. Parallel String Matching. In \textit{European Conference on Parallel Processing} (pp. 187-198). Springer, Cham.
\bibitem{psysoc-DOI}Faro, S., 2016, June. Evaluation and improvement of fast algorithms for exact matching on genome sequences. In \textit{International Conference on Algorithms for Computational Biology} (pp. 145-157). Springer, Cham.
\bibitem{psysoc-DOI}Oladunjoye, J.A., Afolabi, A.O., Olabiyisi, S.O. and Moses, T., 2017. A Comparative Analysis of Pattern Matching Algorithm Using Bit-Parallelism Technique. \textit{IUP Journal of Information Technology}, 13(4), pp.20-36.

\bibitem{psysoc-DOI}Fredriksson, K., 2003. Shift-or string matching with super-alphabets. Information Processing Letters, 87(4), pp.201-204.
\bibitem{psysoc-DOI}Kida, T., Takeda, M., Shinohara, A. and Arikawa, S., 1999, July. Shift-And approach to pattern matching in LZW compressed text. In \textit{Annual Symposium on Combinatorial Pattern Matching (pp. 1-13)}. Springer, Berlin, Heidelberg.
\bibitem{psysoc-DOI}Goyal, R. and Billa, S.L., Cavium Inc, 2017. Generating a non-deterministic finite automata (NFA) graph for regular expression patterns with advanced features. U.S. Patent
9,563,399.

\bibitem{psysoc-DOI}Hirvola, T. and Tarhio, J., 2017. Bit-Parallel Approximate Matching of Circular Strings with k Mismatches. \textit{Journal of Experimental Algorithmics (JEA)}, 22, pp.1-5.

\bibitem{psysoc-DOI}Alfred, V., 2014. Algorithms for finding patterns in strings. Algorithms and Complexity,p.255, pp.1-5.

\end{thebibliography}

\end{document}
